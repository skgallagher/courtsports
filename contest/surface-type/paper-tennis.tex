\documentclass[]{article}
\usepackage{lmodern}
\usepackage{amssymb,amsmath}
\usepackage{ifxetex,ifluatex}
\usepackage{fixltx2e} % provides \textsubscript
\ifnum 0\ifxetex 1\fi\ifluatex 1\fi=0 % if pdftex
  \usepackage[T1]{fontenc}
  \usepackage[utf8]{inputenc}
\else % if luatex or xelatex
  \ifxetex
    \usepackage{mathspec}
  \else
    \usepackage{fontspec}
  \fi
  \defaultfontfeatures{Ligatures=TeX,Scale=MatchLowercase}
\fi
% use upquote if available, for straight quotes in verbatim environments
\IfFileExists{upquote.sty}{\usepackage{upquote}}{}
% use microtype if available
\IfFileExists{microtype.sty}{%
\usepackage{microtype}
\UseMicrotypeSet[protrusion]{basicmath} % disable protrusion for tt fonts
}{}
\usepackage[margin=1in]{geometry}
\usepackage{hyperref}
\hypersetup{unicode=true,
            pdftitle={Analyzing the Effect of Surface in Tennis Grand Slams},
            pdfauthor={Kayla Frisoli, Shannon Gallagher, and Amanda Luby},
            pdfborder={0 0 0},
            breaklinks=true}
\urlstyle{same}  % don't use monospace font for urls
\usepackage{graphicx,grffile}
\makeatletter
\def\maxwidth{\ifdim\Gin@nat@width>\linewidth\linewidth\else\Gin@nat@width\fi}
\def\maxheight{\ifdim\Gin@nat@height>\textheight\textheight\else\Gin@nat@height\fi}
\makeatother
% Scale images if necessary, so that they will not overflow the page
% margins by default, and it is still possible to overwrite the defaults
% using explicit options in \includegraphics[width, height, ...]{}
\setkeys{Gin}{width=\maxwidth,height=\maxheight,keepaspectratio}
\IfFileExists{parskip.sty}{%
\usepackage{parskip}
}{% else
\setlength{\parindent}{0pt}
\setlength{\parskip}{6pt plus 2pt minus 1pt}
}
\setlength{\emergencystretch}{3em}  % prevent overfull lines
\providecommand{\tightlist}{%
  \setlength{\itemsep}{0pt}\setlength{\parskip}{0pt}}
\setcounter{secnumdepth}{5}
% Redefines (sub)paragraphs to behave more like sections
\ifx\paragraph\undefined\else
\let\oldparagraph\paragraph
\renewcommand{\paragraph}[1]{\oldparagraph{#1}\mbox{}}
\fi
\ifx\subparagraph\undefined\else
\let\oldsubparagraph\subparagraph
\renewcommand{\subparagraph}[1]{\oldsubparagraph{#1}\mbox{}}
\fi

%%% Use protect on footnotes to avoid problems with footnotes in titles
\let\rmarkdownfootnote\footnote%
\def\footnote{\protect\rmarkdownfootnote}

%%% Change title format to be more compact
\usepackage{titling}

% Create subtitle command for use in maketitle
\newcommand{\subtitle}[1]{
  \posttitle{
    \begin{center}\large#1\end{center}
    }
}

\setlength{\droptitle}{-2em}

  \title{Analyzing the Effect of Surface in Tennis Grand Slams}
    \pretitle{\vspace{\droptitle}\centering\huge}
  \posttitle{\par}
    \author{Kayla Frisoli, Shannon Gallagher, and Amanda Luby}
    \preauthor{\centering\large\emph}
  \postauthor{\par}
      \predate{\centering\large\emph}
  \postdate{\par}
    \date{September 12, 2018}

\usepackage{booktabs}
\usepackage{longtable}
\usepackage{array}
\usepackage{multirow}
\usepackage[table]{xcolor}
\usepackage{wrapfig}
\usepackage{float}
\usepackage{colortbl}
\usepackage{pdflscape}
\usepackage{tabu}
\usepackage{threeparttable}
\usepackage{threeparttablex}
\usepackage[normalem]{ulem}
\usepackage{makecell}

\usepackage{hyperref}

\begin{document}
\maketitle
\begin{abstract}
Tennis grand slams consist of the Australian Open, French Open,
Wimbledon, and US Open, which are played on hard (Plexicushion), clay,
grass, and hard (DecoTurf) courts, respectively. The surface type may
substantially impact ball speed, height, and spin as well as player
speed and agility. It is also believed that play style and practice
habits may contribute to different results across surface types. For
example, Rafael Nadal is thought to be the best clay court player of all
time whereas Roger Federer is particularly known for dominance at
Wimbledon. On the women's side, Serena Williams once struggled on clay
courts but has seemingly transformed her style to perform better on clay
courts, but has perhaps suffered on grass as a consequence. In this
analysis, we examine the result of the top 100 players in grand slams
from 2013-2017 across the four different surfaces. We create a
hierarchical model with fixed and random effects to predict the number
of points won in a match. We take into consideration player-specific
effects, nationality (which is thought to have an effect on play style),
sex, ranking, ELO, and game statistics. We assess the fit of our model
using standard statistical techniques (e.g.~MSE, AIC, BIC, residual
diagnostics) in addition to `common knowledge' factors (for instance,
Rafael Nadal should be indicated as a superior clay court player by the
model). We compare the results of top 100 players across grand slams to
examine the effect of court surface. We also provide an in-depth
analysis of Nadal, Federer, and S. Williams.
\end{abstract}

\hypertarget{sec:iintro}{%
\section{Introduction}\label{sec:iintro}}

Rafael Nadal is known as the ``King of Clay'' in tennis, having won 11
out of his current 17 grand slams titles at the French Open, which is
played on a clay surface (Jurejko 2018). In contrast, his rival Roger
Federer has won his most grand slam titles (8 out of 20) at Wimbledon,
which is played on grass. On the women's side, Serena Williams has been
dominant both on hard court (7 titles at the Australian Open and 6 at US
Open) and grass (7 at Wimbledon). This trend extends to other top
players, who seem to have better results at some grand slams than
others. More broadly, it seems that country of origin has an interaction
effect with court type. For example, Spanish players seem to excel on
clay courts and Americans have great success at Wimbledon despite grass
courts not being of wide use in the USA. It also worth questioning
whether the US Open and Australian Open should be grouped together as
hard courts despite having different surface compositions (Paxinos
2007). In this paper, we analyze the results of grand slam players from
2013-2017, and we

\begin{enumerate}
\def\labelenumi{\arabic{enumi}.}
\item
  Determine if and how court surface effects players by implementation
  of a series of nested hierarchical models
\item
  Examine how Nadal, Federer, and Williams' play differs by surface
\item
  Assess whether we can group the two hard court surfaces together.
\end{enumerate}

As to issue (1) quantifying the effect of court surface on players,
there has not been much written about with regards to tennis. There are
materials available in the literature for forecasting the outcome of
tennis matches (Klaassen and Magnus 2003; Newton and Keller 2005; McHale
and Morton 2011; Kovalchik 2016){]} or for assessing whether points
within a match are independent and identically distributed (Klaassen and
Magnus 2001). (Knottenbelt, Spanias, and Madurska 2012) do take into
account surface in their model but do not compare the results of one
surface to another. Other sports analyses do take into account surface
type such as grass vs.~turf in soccer and football. Results from these
studies show that surface type does have an effect on the game, either
directly or indirectly {[}Andersson, Ekblom, and Krustrup (2008); Gains
et al. (2010);{]}.

We use models that take into account both individual and group effects
such as in the Gaussian-process player production basketball model or
predicting individual soccer performance (Page, Barney, and McGuire
2013; Egidi and Gabry 2018). Both of those models had success using
hierarhcical Bayesian models, which we employ in our own models. More
specifically, we model the players' expected points in a match based on
the player's own characteristics, the court/tournament effects, and the
oponent's ranking.

For issue (2) the player analysis of Nadal, Federer, and Williams, we
examine whether our model passes the ``common sense'' tests like how the
models in {[}thomas2013{]} show that commonly well known hockey players
also have high status in the model. We also examine whether these
players do have surface apparent effects. Few academic papers have been
written about Nadal, Federer, or Williams. One paper studies Federer's
odds of winning when Nadal suddenly withdrew from Wimbledon showed that
Federer was too heavily favored by bookmakers (Leitner, Zeileis, and
Hornik 2009). One analysis of Williams shows how she has gotten better
with age, even past the point when other greats began to decline, but
the study does not look at surface type (Morris 2015).

Finally, for issue (3), we use clustering methods {[}cite{]} in order to
determine which court surface types are more similar to one another.

Readers may object that we are looking at differences between grand
slams, which each have their own time period, weather conditions, play
time conditions, and ``home court effects'' instead of differences in
surfaces alone. However, (1) grand slam data is the most readily
available and most complete which makes it the best choice at the moment
for modelling, (2) we adjust for these confounders where we can, and (3)
analyzing the difference in the grand slams is still useful as they are
considered to be the most prestigious events in tennis.

The rest of this paper is organized as follows. In
\protect\hyperlink{sec:data}{Section Data} we describe our grand slam
tennis data. In \protect\hyperlink{sec:eda}{Section Early Data Analysis}
we examine the data at a high level and use clustering whether to
determine how the courts differ from one another. In
\protect\hyperlink{sec:methods}{Section Methods} we describe our
hierarchical models we use to determine difference in court surfaces. In
\protect\hyperlink{sec:results}{Section Results} we describe the results
of our modelling and also examine the play of Nadal, Federer, and
Williams. Finally in \protect\hyperlink{discussion}{Section Discussion},
we discuss future work and extensions or our model.

\hypertarget{sec:data-eda}{%
\section{Data and EDA}\label{sec:data-eda}}

\hypertarget{sec:data}{%
\subsection{Data}\label{sec:data}}

The data consists of 5080 matches split evenly over the four grand slams
and the two leagues: ATP (men's) and WTA (women's). Each match has 80
attributes, many of which are redundant. We focus on the following
attributes for both the winner and loser of the match: games won, points
won, retirement, break points faced, break points saved, aces, country
of origin, and player attributes. Additionally, we take into account the
number of sets in a match, the surface type, minutes played, and round
of the tournament. A subset of the data is shown in Table
\ref{tab:data}.

\rowcolors{2}{gray!6}{white}
\begin{table}

\caption{\label{tab:tab-data}\label{tab:data}Example of the grand slam data.  It includes winner and loser attributes, match attributes, and tournament attributes.  Not all attributes are shown here.}
\centering
\begin{tabular}[t]{llrlrrrr}
\hiderowcolors
\toprule
Winner & Tournament & Year & W. IOC & W. Points & W. Rank & L. Points & L. Rank\\
\midrule
\showrowcolors
Serena Williams & Australian Open & 2013 & USA & 52 & 3 & 18 & 110\\
Serena Williams & Australian Open & 2013 & USA & 70 & 3 & 41 & 112\\
Roger Federer & Australian Open & 2013 & SUI & 95 & 2 & 63 & 46\\
Roger Federer & Australian Open & 2013 & SUI & 111 & 2 & 86 & 40\\
Rafael Nadal & Roland Garros & 2013 & ESP & 140 & 4 & 115 & 59\\
Rafael Nadal & Roland Garros & 2013 & ESP & 113 & 4 & 90 & 35\\
\bottomrule
\end{tabular}
\end{table}
\rowcolors{2}{white}{white}

The data is obtained from Jeff Sackman's open website {[}cite{]} via the
\texttt{R} package \texttt{deuce} {[}cite{]}. All steps of our analysis
from collection to dissemination are freely available online and are
compiled into an \texttt{R} package
\url{http://github.com/shannong19/courtsports}.

\hypertarget{sec:eda}{%
\subsection{Early Data Analysis}\label{sec:eda}}

\hypertarget{sec:methods}{%
\section{Methods}\label{sec:methods}}

\hypertarget{sec:results}{%
\section{Results}\label{sec:results}}

\hypertarget{sec:discussion}{%
\section{Discussion}\label{sec:discussion}}

\hypertarget{sec:refs}{%
\section*{References}\label{sec:refs}}
\addcontentsline{toc}{section}{References}

\hypertarget{refs}{}
\leavevmode\hypertarget{ref-andersson2008}{}%
Andersson, Helena, Björn Ekblom, and Peter Krustrup. 2008. ``Elite
Football on Artificial Turf Versus Natural Grass: Movement Patterns,
Technical Standards, and Player Impressions'' 26 (February): 113--22.

\leavevmode\hypertarget{ref-egidi2018}{}%
Egidi, Leonardo, and Jonah Gabry. 2018. ``Bayesian Hierarchical Models
for Predicting Individual Performance in Soccer.'' \emph{Journal of
Quantitative Analysis in Sports} 14 (3). De Gruyter: 143--57.

\leavevmode\hypertarget{ref-gains2010}{}%
Gains, Graydon L, Andy N Swedenhjelm, Jerry L Mayhew, H Michael Bird,
and Jeremy J Houser. 2010. ``Comparison of Speed and Agility Performance
of College Football Players on Field Turf and Natural Grass.'' \emph{The
Journal of Strength \& Conditioning Research} 24 (10). LWW: 2613--7.

\leavevmode\hypertarget{ref-bbc2018}{}%
Jurejko, Jonathan. 2018. ``French Open 2018: Why does 'King of Clay'
Rafael Nadal reign supreme?'' Edited by BBC Sport at Roland Garros.
\url{https://www.bbc.com/sport/tennis/44385223}.

\leavevmode\hypertarget{ref-klaassen2003}{}%
Klaassen, Franc J.G.M., and Jan R. Magnus. 2003. ``Forecasting the
Winner of a Tennis Match.'' \emph{European Journal of Operational
Research} 148 (2): 257--67.
\url{https://doi.org/https://doi.org/10.1016/S0377-2217(02)00682-3}.

\leavevmode\hypertarget{ref-klaassen2001}{}%
Klaassen, Franc J.G.M., and Jan R Magnus. 2001. ``Are Points in Tennis
Independent and Identically Distributed? Evidence from a Dynamic Binary
Panel Data Model.'' \emph{Journal of the American Statistical
Association} 96 (454). Taylor \& Francis: 500--509.
\url{https://doi.org/10.1198/016214501753168217}.

\leavevmode\hypertarget{ref-knottenbelt2012}{}%
Knottenbelt, William J., Demetris Spanias, and Agnieszka M. Madurska.
2012. ``A Common-Opponent Stochastic Model for Predicting the Outcome of
Professional Tennis Matches.'' \emph{Computers \& Mathematics with
Applications} 64 (12): 3820--7.
\url{https://doi.org/https://doi.org/10.1016/j.camwa.2012.03.005}.

\leavevmode\hypertarget{ref-kovalchik2016}{}%
Kovalchik, Stephanie Ann. 2016. ``Searching for the Goat of Tennis Win
Prediction.'' \emph{Journal of Quantitative Analysis in Sports} 12 (3).
De Gruyter: 127--38.

\leavevmode\hypertarget{ref-leitner2009}{}%
Leitner, Christoph, Achim Zeileis, and Kurt Hornik. 2009. ``Is Federer
Stronger in a Tournamentwithout Nadal? An Evaluation of Odds and
Seedings for Wimbledon 2009.'' \emph{Austrian Journal of Statistics} 38
(4): 277--86.

\leavevmode\hypertarget{ref-mchale2011}{}%
McHale, Ian, and Alex Morton. 2011. ``A Bradley-Terry Type Model for
Forecasting Tennis Match Results.'' \emph{International Journal of
Forecasting} 27 (2): 619--30.
\url{https://doi.org/https://doi.org/10.1016/j.ijforecast.2010.04.004}.

\leavevmode\hypertarget{ref-five2015}{}%
Morris, Benjamin. 2015. ``Serena Williams and the Difference Between
All-Time Great and Greatest of All Time.'' Edited by FiveThirtyEight.
\url{https://fivethirtyeight.com/features/serena-williams-and-the-difference-between-all-time-great-and-greatest-of-all-time/}.

\leavevmode\hypertarget{ref-newton2005}{}%
Newton, Paul K, and Joseph B Keller. 2005. ``Probability of Winning at
Tennis I. Theory and Data.'' \emph{Studies in Applied Mathematics} 114
(3). Wiley Online Library: 241--69.

\leavevmode\hypertarget{ref-page2013}{}%
Page, Garritt L, Bradley J Barney, and Aaron T McGuire. 2013. ``Effect
of Position, Usage Rate, and Per Game Minutes Played on Nba Player
Production Curves.'' \emph{Journal of Quantitative Analysis in Sports} 9
(4). De Gruyter: 337--45.

\leavevmode\hypertarget{ref-theage2007}{}%
Paxinos, Stathi. 2007. ``Australian Open court surface is speeding up.''
Edited by The Age.
\url{https://www.theage.com.au/sport/tennis/australian-open-court-surface-is-speeding-up-20071120-ge6chj.html}.


\end{document}
